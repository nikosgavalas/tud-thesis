%!TEX root = ../main.tex

% Appendix Template

\chapter{Code} % Main appendix title
\label{Appendix-A-code} % Change X to a consecutive letter; for referencing this appendix elsewhere, use \ref{AppendixX}

\section{Key-value store API}

\begin{lstlisting}[language=Python, caption=Key-value store API - method signatures.]
class KVStore:
    def __getitem__(self, key: bytes) -> bytes:
        ...

    def __setitem__(self, key: bytes, value: bytes) -> None:
        ...

    def get(self, key: bytes) -> bytes:
        ...

    def set(self, key: bytes, value: bytes) -> None:
        ...

    def __sizeof__(self) -> int:
        ...

    def close(self) -> None:
        ...

    def snapshot(self, id: int) -> None:
        ...

    def restore(self, version: Optional[int] = None) -> None:
        ...
\end{lstlisting}

\section{Remote API}

\begin{lstlisting}[language=Python, caption=Remote API - method signatures.]
class Remote:
    def put(self, filename: str) -> None:
        ...

    def get(self, filename: str) -> None:
        ...

    def gc(self) -> None:
        ...

    def restore(self, version: Optional[int] = None) -> None:
        ...

    def destroy(self) -> None:
        ...
\end{lstlisting}

% \begin{lstlisting}[language=Python, caption=Algorithm used in snapshot rollback.]
% def expand_version(version: int, max_per_level: int) -> list[tuple[int, int]]:
%     acc = []
%     while version != 0:
%         acc.append(version % max_per_level)
%         version //= max_per_level

%     levels_runs = []
%     for i, e in enumerate(acc):
%         j = e - 1
%         while j >= 0:
%             levels_runs.append((i, j))
%             j -= 1

%     return levels_runs
% \end{lstlisting}
