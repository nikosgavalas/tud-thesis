%!TEX root = ../main.tex

% Appendix Template

\chapter{Code} % Main appendix title

\label{Appendix-A-code} % Change X to a consecutive letter; for referencing this appendix elsewhere, use \ref{AppendixX}

\section{Key-value store API}

\begin{lstlisting}[language=Python, caption=Key-value store API - method signatures.]
class KVStore:
    def __getitem__(self, key: bytes) -> bytes:
        pass

    def __setitem__(self, key: bytes, value: bytes) -> None:
        pass

    def get(self, key: bytes) -> bytes:
        pass

    def set(self, key: bytes, value: bytes) -> None:
        pass

    def __sizeof__(self) -> int:
        pass

    def close(self) -> None:
        pass

    def snapshot(self) -> None:
        pass

    def restore(self, version: Optional[int] = None) -> None:
        pass
\end{lstlisting}

\section{Replica API}

\begin{lstlisting}[language=Python, caption=Replica API - method signatures.]
class Replica:
    def put(self, filename: str) -> None:
        pass

    def get(self, filename: str, version: int = None):
        pass

    def gc(self):
        pass

    def restore(self, max_per_level: int, version: int = None):
        pass

    def destroy(self):
        pass
\end{lstlisting}

\begin{lstlisting}[language=Python, caption=Algorithm used in snapshot rollback.]
def expand_version(version: int, max_per_level: int) -> list[tuple[int, int]]:
    acc = []
    while version != 0:
        acc.append(version % max_per_level)
        version //= max_per_level

    levels_runs = []
    for i, e in enumerate(acc):
        j = e - 1
        while j >= 0:
            levels_runs.append((i, j))
            j -= 1

    return levels_runs
\end{lstlisting}
