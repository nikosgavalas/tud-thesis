%!TEX root = ../main.tex

\chapter{Evaluation}

\label{Chapter4-evaluation}

% TODO write some intro here

\section{Parameters}

Each of our implemented key-value stores is instantiated with a set of parameters. In Chapter \ref{Chapter3-implementation} we explained what each parameter represents, but to be able to understand the trade-offs among them, and how various settings of them influence the behaviour of the respective engine, it is important to explore them visually.

\subsection{Log-structured Merge-Tree}

The first parameter of the LSM-Tree is \verb"max_runs_per_level". This controls the maximum amount of runs allowed in a level when log-structuring and will be more carefully examined in the subsection \ref{subsection-max-runs-per-level} as it is relevant to the other engines as well.

\subsubsection{Density Factor}

The \verb"density_factor", as explained in section \ref{subsection-lsm-design}, controls the width of gaps between the fence pointers of the LSM-Tree.



% TODO show density factor, read throughput vs memory/diskspace

% TODO memtable bytes limit, read throughput vs memory usage

\subsection{HybridLog}

% TODO show mem_seg_length, read/write throughput vs memory

% TODO show ro_lag_interval, read/write throughput vs memory

% TODO compaction, does it affect the system?

\subsection{AppendLog}

% TODO threshold

\subsection{Number of runs per level in Log-structuring}
\label{subsection-max-runs-per-level}

% TODO show runs per level, how influences read-write perf.
% with rpl up, write throughput should go up, reads down.

\section{Comparison}

% for uniform distribution:
%   - write latency vs throughput, 3 engs, 50p/95p
%   - read latency vs throughput, 3 engs, 50p/95p
%   - keep track of memory as well, 3engs

% then for zipfian, same

\section{Incremental Snapshotting}

% show snapshot time vs state size
