%!TEX root = ../main.tex

\chapter{Related Work} % Main chapter title

\label{Chapter2-related-work}

\section{TODO}

% related work: https://www.sciencedirect.com/science/article/pii/S0306437922000229?ref=pdf_download&fr=RR-2&rr=7c0f0989de86b7af

% faster

% LSM tree -> cassandra/rocksdb

% find some paper for bitcask kvstore

% fractal trees

% In support of workload-aware streaming state management

% JSON crdt?

% Lightweight Asynchronous Snapshots for Distributed Dataflows

%% **** state management in apache flink section 4.2!!

% for transactional sfaas check the transactions across serverless functions leveraging.. .paper




% There are several types of disk-based key-value stores available, including:

% LevelDB: A popular open-source key-value store that is optimized for read-heavy workloads and has a small memory footprint.

% RocksDB: A fork of LevelDB that is designed to handle a wider range of workloads, including write-heavy ones.

% LMDB: A high-performance, memory-mapped key-value store that supports multi-threading and transactions.

% Cassandra: A distributed key-value store that is optimized for write-heavy workloads and provides tunable consistency levels.

% Riak: A distributed key-value store that is designed for high availability and fault-tolerance and supports both eventual and strong consistency.

% Berkeley DB: A mature and battle-tested embedded key-value store that supports transactions, replication, and high availability.

% Kyoto Cabinet: An efficient and lightweight key-value store that is designed for high-speed data storage and retrieval.

% Redis: A popular in-memory key-value store that also provides disk-based persistence, making it suitable for large datasets that cannot fit in memory.




% TODO - argue why you selected these databases, LSMT paper shits on B-trees so use that info from there



% NOTE - including 2 images side by side before i forget

% \begin{figure}[H]
%     \begin{subfigure}{.5\textwidth}
%         \centering
%         \includegraphics[width=0.8\linewidth]{1a.png}
%         \caption{NN}
%         \label{fig:1a}
%     \end{subfigure}
%     \begin{subfigure}{.5\textwidth}
%         \centering
%         \includegraphics[width=0.8\linewidth]{1b.png}
%         \caption{Tri-linear}
%         \label{fig:1b}
%     \end{subfigure}
%     \caption{Tri-linear vs Nearest Neighbours Interpolation}
%     \label{fig:1}
% \end{figure}

% inserting images in general: https://www.overleaf.com/learn/latex/Inserting_Images

% Note, instead of B-trees there are also on disk hash-tables, like gdbm.
%

% see the intensive data book, explain write amplification.
