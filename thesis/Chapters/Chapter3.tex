%!TEX root = ../main.tex

\chapter{Implementation}

\label{Chapter3-implementation}

In this chapter, we will present some high-level design decisions that are common in all our implementations, then we will go through the internals and the implementation details of each of the key-value (KV) stores and finally present we leveraged log-structuring to fullfil the incremental snapshotting property we want our key-value store to have.

\section{Common design decisions}

Firstly, we designed our implementations to expose a common interface (API) to the user.
This allows for easy benchmarking, testing, and ultimately a fair comparison between the engines.
The API programmatically is defined in a parent class that the classes corresponding to each engines inherit and extend.
Firstly, all the engines have a common API. This can be found in appendix \ref{Appendix-A-code}.


1. empty value == delet
2. binary keys and values because allows us to encode the length first for the encoding
3. disk binary encoding
4. data under a single directory
5. engine will rebuild from local files if found. if given remote replica, will
restore the latest version by default. should be able to rollback

\section{Log-Structured Merge-Tree}

% maybe move these to related work??

The Log-Structured Merge-Tree (LSM-Tree) is a disk-based data structure [\cite{lsmtree}], and one of the most prominent, battle-tested, and well-researched key-value store backend engines.
It was invented by Patrick O'Neil in 1996 and has since been used in multiple databases, such as Google's LevelDB [\cite{leveldb}], Meta's RocksDB [\cite{rocksdb}] and Apache's Cassandra [\cite{cassandra}].

The LSM-Tree makes extensive use of the \textit{log-structuring} technique, which first appeared in the LFS file system [\cite{lsm-filesystem}] and has since been used not only in LSM-Tree-based database management systems, but also in other types of storage engines, even B-Tree-based ones [\cite{llama}].

Log-structuring offers significant speedups by significantly reducing the number of writes per page and transforming them into a "sequential" format.
In other words, it consolidates numerous random writes into a single large multi-page write [\cite{llama}].

In this work, we use log-structuring extensively, because, besides its advantages in I/O operations, it also provides a straightforward way to create incremental snapshots of the database's state.
We analyze the way we leveraged log-structuring for incremental snapshotting later, in section \ref{section-incremental-snapshots}.

Given the close relationship between log-structuring and the LSM-Tree (which makes extensive use of it), we will introduce the concept in tandem with the LSM-Tree.

\subsection{Design}

The power of the LSM-Tree can be partially attributed to the fact that it uses lightweight indices, when compared to B-trees which effectively double the cost of every I/O operation to maintain their indices [\cite{lsmtree}].
This enables the LSM-Tree to scale to very high write and read rates.

However, one other important factor for the LSM-Tree's fast I/O is the use of an in-memory buffer, which aggregates the updates and when it's full, it flushes them to disk sequentially.
As it is well known, disks perform much faster sequential operations that operations than require random-access, especially in the cloud, where inexpensive disks have limited I/O rates [\cite{llama}].

This buffer flushes the aggregated data into \textit{sorted} chunks of data that are commonly referred to as SSTs for ``Sorted String Tables'', but we will just call them ``runs''.
Sorting is essential for indexing, as it enables us to lookup keys in logarithmic time.

So, initially, as we are writing data, we keep them in our buffer, and when this buffer is full, we flush it into a file that we call a run.

% LSMT with size-tiered compaction (write-optimized)

% explain how it works

% possible optimizations:
% 1. https://www.youtube.com/watch?v=b6SI8VbcT4w
% 2. fence pointers can be used to organize data into compressible blocks
% 3. the .filter and .pointers files could be embedded to the runfile (and then do a relative seek from the end)
% this would help with the save to replica things


\section{Append Log}



\section{Hybrid Log}

% possible extensions/optimizations applicable to all engines:
% concurrency! explain how it can be applied, what to lock etc.
% mmap for the files
% checksums for malformed records

\section{Incremental Snapshots}
\label{section-incremental-snapshots}
